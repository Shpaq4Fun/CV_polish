%%%%%%%%%%%%%%%%%
% This is an sample CV template created using altacv.cls
% (v1.1.4, 27 July 2018) written by LianTze Lim (liantze@gmail.com). Now compiles with pdfLaTeX, XeLaTeX and LuaLaTeX.
% 
%% It may be distributed and/or modified under the
%% conditions of the LaTeX Project Public License, either version 1.3
%% of this license or (at your option) any later version.
%% The latest version of this license is in
%%    http://www.latex-project.org/lppl.txt
%% and version 1.3 or later is part of all distributions of LaTeX
%% version 2003/12/01 or later.
%%%%%%%%%%%%%%%%

%% If you need to pass whatever options to xcolor
\PassOptionsToPackage{dvipsnames}{xcolor}

%% If you are using \orcid or academicons
%% icons, make sure you have the academicons 
%% option here, and compile with XeLaTeX
%% or LuaLaTeX.
% \documentclass[10pt,a4paper,academicons]{altacv}

%% Use the "normalphoto" option if you want a normal photo instead of cropped to a circle
% \documentclass[10pt,a4paper,normalphoto]{altacv}

\documentclass[10pt,a4paper]{altacv}
%% AltaCV uses the fontawesome and academicon fonts
%% and packages. 
%% See texdoc.net/pkg/fontawecome and http://texdoc.net/pkg/academicons for full list of symbols.

%% 
%% Compile with LuaLaTeX for best results. If you
%% want to use XeLaTeX, you may need to install
%% Academicons.ttf in your operating system's font 
%% folder.


% Change the page layout if you need to
\geometry{left=1cm,right=9cm,marginparwidth=6.8cm,marginparsep=1.2cm,top=1.25cm,bottom=1.25cm,footskip=2\baselineskip}

% Change the font if you want to.

% If using pdflatex:
% \usepackage[T1]{fontenc}
% \usepackage[utf8]{inputenc}
% \usepackage[default]{lato}

% If using xelatex or lualatex:
\setmainfont{Lato}
\usepackage{academicons}
% Change the colours if you want to
\definecolor{Mulberry}{HTML}{72243D}
\definecolor{SlateGrey}{HTML}{2E2E2E}
\definecolor{LightGrey}{HTML}{666666}
\colorlet{heading}{Sepia}
\colorlet{accent}{Mulberry}
\colorlet{emphasis}{SlateGrey}
\colorlet{body}{LightGrey}

% Change the bullets for itemize and rating marker
% for \cvskill if you want to
\renewcommand{\itemmarker}{{\small\textbullet}}
\renewcommand{\ratingmarker}{\faCircle}
%% sample.bib contains your publications
\addbibresource{sample.bib}

\usepackage[colorlinks]{hyperref}

\begin{document}
\tagline{Curriculum Vitae }
\name{JACEK WODECKI}
\tagline{Curriculum Vitae}
\photo{3cm}{indeks}
\personalinfo{%
  % Not all of these are required!
  % You can add your own with \printinfo{symbol}{detail}
  \email{wodecki.jacek@gmail.com }
  \phone{515 868 995}
  \mailaddress{ul. Maślicka 177d/4, 54-104 Wrocław}
  \location{Wrocław, Polska}
  \homepage{http://labdiag.pwr.wroc.pl/jwod}
  \linkedin{linkedin.com/in/jacek-wodecki-0b5290b2}
  %% You MUST add the academicons option to \documentclass, then compile with LuaLaTeX or XeLaTeX, if you want to use \orcid or other academicons commands.
   \orcid{orcid.org/0000-0002-3163-8678}
   \aiResearchGate{researchgate.net/profile/Jacek\_Wodecki}
%   \aiGoogleScholar{https://scholar.google.pl/citations?user=y9bqYcUAAAAJ&hl=pl}
}

%% Make the header extend all the way to the right, if you want. 
\begin{fullwidth}
\makecvheader
\end{fullwidth}

%% Depending on your tastes, you may want to make fonts of itemize environments slightly smaller
% \AtBeginEnvironment{itemize}{\small}


%% Provide the file name containing the sidebar contents as an optional parameter to \cvsection.
%% You can always just use \marginpar{...} if you do
%% not need to align the top of the contents to any
%% \cvsection title in the "main" bar.
\cvsection[page1sidebar]{Doświadczenie}

\cvevent{Specjalista inżynieryjno-techniczny}{KGHM Cuprum Centrum Badawczo-Rozwojowe Sp. z o.o.}{Sierpień 2015 -- Obecnie}{Wrocław, Polska}
\begin{itemize}
    \item Analityka danych przemysłowych o znaczeniu diagnostycznym
    \item Przetwarzanie sygnałów cyfrowych
    \item Projektowanie algorytmów na potrzeby diagnostyki maszyn górniczych
    \item Implementacja algorytmów diagnostycznych w środowisku Matlab
    \item Tworzenie raportów i dokumentacji technicznych
    \item Udział w projektach komercyjnych (np. dla KGHM) i dotowanych (np. Horyzont2020, KIC)
\end{itemize}

\divider

\cvevent{Stażysta}{Wrocławskie Centrum Badań EIT+}{Lipiec 2014 -- Czerwiec 2015}{Wrocław, Polska}
\begin{itemize}
    \item Modelowanie laserowych procesów dyspersyjnych 
    \item Przetwarzanie sygnałów cyfrowych
    \item Implementacja algorytmów analitycznych w środowisku Matlab
    \item Projektowanie i przeprowadzanie eksperymentów laboratoryjnych
\end{itemize}

% \cvsection{Projects}

% \cvevent{Project 1}{Funding agency/institution}{Project duration}{}
% \begin{itemize}
% \item Details
% \end{itemize}

% \divider

% \cvevent{Project 2}{Funding agency/institution}{Project duration}{}
% A short abstract would also work.

\medskip

% \cvsection{A Day of My Life}

% % Adapted from @Jake's answer from http://tex.stackexchange.com/a/82729/226
% % \wheelchart{outer radius}{inner radius}{
% % comma-separated list of value/text width/color/detail}
% \wheelchart{1.5cm}{0.5cm}{%
%   6/8em/accent!30/{Sleep,\\beautiful sleep}, 
%   4/8em/accent!8/{Watching news},
%   3/7em/accent!8/Meeting &Talking with friends,
%   8/8em/accent!55/Day time Job,
%   2/10em/accent!10/Sports and relaxation,
%   5/6em/accent!20/Spending time with family
% }

% \clearpage
\cvsection{Wybrane publikacje}

\nocite{*}

% \printbibliography[heading=pubtype,title={\printinfo{\faBook}{Books}},type=book]

% \divider

\printbibliography[heading=pubtype,title={\printinfo{\faFileTextO}{Artykuły w czasopismach naukowych}},type=article]

% \divider

% \printbibliography[heading=pubtype,title={\printinfo{\faGroup}{Conference Proceedings}},type=inproceedings]

%% If the NEXT page doesn't start with a \cvsection but you'd
%% still like to add a sidebar, then use this command on THIS
%% page to add it. The optional argument lets you pull up the 
%% sidebar a bit so that it looks aligned with the top of the
%% main column.
% \addnextpagesidebar[-1ex]{page3sidebar}

\end{document}
